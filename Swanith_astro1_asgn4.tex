\documentclass[11pt]{article}

\usepackage{sectsty}
\usepackage{graphicx}
\usepackage{xcolor}
\usepackage{setspace}
\usepackage{hyperref}
\usepackage{gensymb}
\usepackage{amsmath} % for the matrix environments
\usepackage{amsfonts}
\usepackage{amssymb}
\usepackage{cancel}
\usepackage{bbold}
% Margins
\topmargin=-0.90in%-0.45%
\evensidemargin=0in%0in%
\oddsidemargin=0in
\textwidth=7in
\textheight=10.0in
\headsep=0.5in
\onehalfspacing

\title{Astronomy \& Astrophysics Assignment - 4}
\author{\textbf{\Large Swanith Upadhye}}
\date{\today}

\begin{document}
	
	\maketitle
	\noindent\hrulefill
	\Large
	
	\section{\color{teal} Question 1}
	\subsection{(a)Sun collapse, Change in Rotation Period}
	
	The Angular Momentum of Sun, assuming perfectly spherical(uniformly dense) and rotating at constant angular velocity;
	
	\[
		L = I_\odot \omega_\odot = \frac{2}{5} M_\odot R_\odot^2 \omega_\odot
	\]
	Upon collapse;
	\[
		L \propto R^2 \omega
	\]
	Assuming no external torque on Sun (or at rates far smaller than collapse); L remains unchanged:
	\[
		R_\odot^2 \omega_\odot = R^2 \omega
	\]
	(given mass does not change):
	\[
		\Rightarrow \omega = (\frac{R_\odot}{R})^2 \omega_\odot
	\]
	R$_\odot$ =696,340 km\\
	Hence for a collapse into a white dwarf:
	\[
		\omega_{wd} = (\frac{696340}{5000})^2 \times \frac{1}{27} \, 1/days
	\]
	\[
		\Rightarrow \boxed{\tau_{wd} = \frac{1}{\omega_{wd}} \approx 20 \, min}
	\]
	And for collapse into a neutron star:
	\[
		\omega_{ns} = (\frac{696340}{10})^2 \times \frac{1}{27} \, 1/days
	\]
	\[
		\Rightarrow \boxed{\tau_{ns} \approx 0.5 \, msec}
	\]
	which is ultra-relativistic because at any point perpendicular to its rotation axis on its surface has speeds
	\[
		v_{ns-surf} = R_{ns} \omega_{ns} = 10^4 \frac{2\pi}{0.5 \times 10^{-3}} \approx 0.42 c
	\]
	
	\subsection{Neutron Star Disintegration speeds}
	
	Just at disintegration speeds, the Gravitational self energy of the star equals its Rotational Kinetic Energy.\\
	So the Limiting angular speeds(for solid sphere of uniform density and constant angular velocity):
	\[
		\frac{1}{2} I \omega^2 (= \frac{1}{5}MR^2\omega^2)\le \frac{3}{5}\frac{GM^2}{R}
	\]
	\[
		\omega_d \le \sqrt{3\frac{GM}{R^3}}
	\]
	For a Neutron star of 1.4 $M_\odot$ and R = 10 km 
	\[
		\omega_d = \sqrt{\frac{3\times 6.67\times 10^{-11}\times 1.4\times 2\times 10^{30}}{696430000^3}} \approx 0.0041 \, rad \, s^{-1}
	\]
	\[
		\boxed{\tau_d = \frac{2\pi}{\omega_d} \approx 25.5 \,mins}
	\]
	The previous calculation of rotation speeds far exceed this limit. So upon collapse like above sun(now neutron star) must experience mass ejection and radiative energy loss until equlibrium is reached.
	
	\subsection{Deviations from spherical shape}
	
	Given:
	\[
		R_E - R_P = \frac{5\Omega^2 R_{avg}^4}{4GM} = \frac{5\times 0.0041^2 \times 10^{12} }{4\times 6.67 \times 10^{-11} \times 2 \times 10^{30}} 
	\]
	\[
		R_E - R_P \approx 1.57 \times 10^{-13} m
	\]	
	Hence \(\boxed{R_E \approx R_P}\)

	
	\section{\color{teal} Question 2}
	
	\subsection{Pulsar Period Evolution}
	
	Modelling Pulsar as a rotating dipole; Larmor formula gives, radiated power:
	\[
		Power = \frac{dE}{dt} = \frac{\mu_0}{6\pi c^3}|\bar{\ddot{m}}|
	\]
	where $\bar{m}$ is the dipole moment.\\
	Assuming $\bar{m} = \bar{m_0}cos(\omega t) \rightarrow |\bar{\ddot{m}}|^2 \propto \omega^4$
	
	Therefore, \[ Power \propto 1/P^4 \]
	where P is the period of rotation(and c' is a constant).\\
	
	Now for the pulsar the Radiated power is at the cost of the rotational kinetic energy.
	\[
		\frac{dK}{dt} = \frac{d}{dt}( \frac{1}{2}I \omega^2) \propto \frac{d}{dt} (1/P^2) = - \dot{P}/P^3 
	\]
	\[
		\Rightarrow \frac{dK}{dt} \propto - \frac{\dot{P}}{P^3}
	\]
	Hence from the above two relations
	\[
		Power = \frac{dK}{dt} \Rightarrow -\dot{P}P = c......(2.1)
	\]
	where, c is a constant.\\
	
	Integrating the above equation gives:
	\[
		-\frac{P^2}{2} = ct + \frac{P_0^2}{2}
	\] 
	For long times, \(P_0 >> P(t)\) where enough power has been radiated and the pulsar rotates quite slowly compared to its initial speeds;
	\[
		P^2 = 2c \tau
	\]
	Substituting (2.1) into above equation:
	\[
		P^2 = 2(-\dot{P}P)\tau \Rightarrow \boxed{\tau = - \frac{P}{2\dot{P}}} 
	\]
	
	\subsection{Crab Pulsar characteristic lifetime}
	
	For Crab pulsar given P = 0.0331 s and $\dot{P} = 4.22e-13$
	\[
		\tau_C = \frac{0.0331}{2 \times 4.22e-13} \approx \boxed{1243.6 \, years \approx \tau_C}
	\]
	
	\subsection{Power loss of Crab Pulsar}
	
	For M = 1.4 $M_\odot$ and R =10 km, the moment of inertia,
	\[
		I = \frac{2}{5}MR^2 = 0.4 \times 1.4 \times 2 \times 1e30 \times 1e8 = 1.1 \times 10^{38} \, kg \, m^2
	\]
	Power loss
	\[
		\frac{dK}{dt} = \frac{d}{dt} \frac{I(2\pi)^2}{2P^2} = \frac{-4\pi^2 I\dot{P}}{P^3}
	\]
	\[
		\Rightarrow \frac{dK_c}{dt} = \frac{-4\pi^2 (1.1 \times 1e38) \times 4.22 \times 1e-13}{0.0331^3} \approx \boxed{5.05 \times 1e28 \, Watt \approx \frac{dK_c}{dt}}
	\]
	
	\section{\color{teal} Question 3 - Accretion disk Maximum temperature}
	
	Given, the Temperature of Acrretion Disk:
	\[
		T_d(r) = T_0 (R/r)^{3/4}(1-\sqrt{\frac{R}{r}})^{1/4}
	\]
	Taking its derivative and setting it to zero:
	\[
		\frac{dT}{dr} = 0 = \frac{3}{4}(\frac{R}{r})^{-1/4}[R\times(-\frac{1}{r^2})][1-\sqrt{\frac{R}{r}}]^{1/4} + (\frac{R}{r})^{3/4}\frac{1}{4}(1-\sqrt{\frac{R}{r}})^{-3/4}[-\sqrt{R}(-\frac{1}{2}r^{-3/2})]
	\]
	\[
		0 = -\frac{6}{r}(1-\sqrt{\frac{R}{r}}) + \sqrt{\frac{R}{r^3}} 
	\]
	\[
		\Rightarrow \boxed{\sqrt{\frac{r_{max}}{R}} = 7/6}
	\]
	Now,
	\[
		T_{max} = T_0 (36/49)^{3/4}[1-\frac{6}{7}]^{1/4} \approx \boxed{0.488 T_0 \approx T_{d-max}}
	\] 
	
	
\end{document}

