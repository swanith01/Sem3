\documentclass[11pt]{article}

\usepackage{sectsty}
\usepackage{graphicx}
\usepackage{xcolor}
\usepackage{setspace}
\usepackage{hyperref}
\usepackage{gensymb}
\usepackage{amsmath} % for the matrix environments
\usepackage{amsfonts}
\usepackage{amssymb}
\usepackage{cancel}
\usepackage{bbold}
% Margins
\topmargin=-0.90in%-0.45%
\evensidemargin=0in%0in%
\oddsidemargin=0in
\textwidth=7in
\textheight=10.0in
\headsep=0.5in
\onehalfspacing

\title{Astronomy \& Astrophysics Assignment - 4}
\author{\textbf{\Large Swanith Upadhye}}
\date{\today}

\begin{document}
	
	\maketitle
	\noindent\hrulefill
	\Large
	
	\section{\color{teal} Question 1}
	\subsection{(a)Sun collapse, Change in Rotation Period}
	
	The Angular Momentum of Sun, assuming perfectly spherical(uniformly dense) and rotating at constant angular velocity;
	
	\[
		L = I_\odot \omega_\odot = \frac{2}{5} M_\odot R_\odot^2 \omega_\odot
	\]
	Upon collapse;
	\[
		L \propto R^2 \omega
	\]
	Assuming no external torque on Sun (or at rates far smaller than collapse); L remains unchanged:
	\[
		R_\odot^2 \omega_\odot = R^2 \omega
	\]
	(given mass does not change):
	\[
		\Rightarrow \omega = (\frac{R_\odot}{R})^2 \omega_\odot
	\]
	R$_\odot$ =696,340 km\\
	Hence for a collapse into a white dwarf:
	\[
		\omega_{wd} = (\frac{696340}{5000})^2 \times \frac{1}{27} \, 1/days
	\]
	\[
		\Rightarrow \boxed{\tau_{wd} = \frac{1}{\omega_{wd}} \approx 20 \, min}
	\]
	And for collapse into a neutron star:
	\[
		\omega_{ns} = (\frac{696340}{10})^2 \times \frac{1}{27} \, 1/days
	\]
	\[
		\Rightarrow \boxed{\tau_{ns} \approx 0.5 \, msec}
	\]
	which is ultra-relativistic because at any point perpendicular to its rotation axis on its surface has speeds
	\[
		v_{ns-surf} = R_{ns} \omega_{ns} = 10^4 \frac{2\pi}{0.5 \times 10^{-3}} \approx 0.42 c
	\]
	
	\subsection{Neutron Star Disintegration speeds}
	
	Just at disintegration speeds, the Gravitational self energy of the star equals its Rotational Kinetic Energy.\\
	So the Limiting angular speeds(for solid sphere of uniform density and constant angular velocity):
	\[
		\frac{1}{2} I \omega^2 (= \frac{1}{5}MR^2\omega^2)\le \frac{1}{2}\frac{3}{5}\frac{GM^2}{R}
	\]
	(From Virial Theorem):
	\[
		\omega_d \le \sqrt{3\frac{GM}{R^3}}
	\]
	For a Neutron star of 1.4 $M_\odot$ and R = 10 km 
	\[
		\omega_d = \sqrt{\frac{3\times 6.67\times 10^{-11}\times 1.4\times 2\times 10^{30}}{10000^3}} \approx 166492.2 \, rad \, s^{-1}
	\]
	\[
		\boxed{\tau_d = \frac{2\pi}{\omega_d} \approx 0.00038  \,sec}
	\]
	The previous calculation of rotation speeds far exceed this limit. So upon collapse like above sun(now neutron star) must experience mass ejection and radiative energy loss until equlibrium is reached.
	
	\subsection{Deviations from spherical shape}
	
	Given:
	\[
		R_E - R_P = \frac{5\Omega^2 R_{avg}^4}{4GM} = \frac{5\times 166492.2^2 \times 10^{16} }{4\times 6.67 \times 10^{-11} \times 2 \times 10^{30}} 
	\]
	\[
		R_E - R_P \approx 18.4 \, km
	\]	
	Now, from $R_{avg}$
	\[
		R_E + R_P = 20 \, km
	\]
	Hence,
	\[
		\boxed{R_E \approx 19.2 km \, and \, R_P \approx 0.8 km}
	\]
	

	
	\section{\color{teal} Question 2}
	
	\subsection{Pulsar Period Evolution}
	
	Modelling Pulsar as a rotating dipole; Larmor formula gives, radiated power:
	\[
		Power = \frac{dE}{dt} = \frac{\mu_0}{6\pi c^3}|\bar{\ddot{m}}|
	\]
	where $\bar{m}$ is the dipole moment.\\
	Assuming $\bar{m} = \bar{m_0}cos(\omega t) \rightarrow |\bar{\ddot{m}}|^2 \propto \omega^4$
	
	Therefore, \[ Power \propto 1/P^4 \]
	where P is the period of rotation(and c' is a constant).\\
	
	Now for the pulsar the Radiated power is at the cost of the rotational kinetic energy.
	\[
		\frac{dK}{dt} = \frac{d}{dt}( \frac{1}{2}I \omega^2) \propto \frac{d}{dt} (1/P^2) = - \dot{P}/P^3 
	\]
	\[
		\Rightarrow \frac{dK}{dt} \propto - \frac{\dot{P}}{P^3}
	\]
	Hence from the above two relations
	\[
		Power = \frac{dK}{dt} \Rightarrow -\dot{P}P = c......(2.1)
	\]
	where, c is a constant.\\
	
	Integrating the above equation gives:
	\[
		-\frac{P^2}{2} = ct + \frac{P_0^2}{2}
	\] 
	For long times, \(P_0 >> P(t)\) where enough power has been radiated and the pulsar rotates quite slowly compared to its initial speeds;
	\[
		P^2 = 2c \tau
	\]
	Substituting (2.1) into above equation:
	\[
		P^2 = 2(-\dot{P}P)\tau \Rightarrow \boxed{\tau = - \frac{P}{2\dot{P}}} 
	\]
	
	\subsection{Crab Pulsar characteristic lifetime}
	
	For Crab pulsar given P = 0.0331 s and $\dot{P} = 4.22e-13$
	\[
		\tau_C = \frac{0.0331}{2 \times 4.22e-13} \approx \boxed{1243.6 \, years \approx \tau_C}
	\]
	
	\subsection{Power loss of Crab Pulsar}
	
	For M = 1.4 $M_\odot$ and R =10 km, the moment of inertia,
	\[
		I = \frac{2}{5}MR^2 = 0.4 \times 1.4 \times 2 \times 1e30 \times 1e8 = 1.1 \times 10^{38} \, kg \, m^2
	\]
	Power loss
	\[
		\frac{dK}{dt} = \frac{d}{dt} \frac{I(2\pi)^2}{2P^2} = \frac{-4\pi^2 I\dot{P}}{P^3}
	\]
	\[
		\Rightarrow \frac{dK_c}{dt} = \frac{-4\pi^2 (1.1 \times 1e38) \times 4.22 \times 1e-13}{0.0331^3} \approx \boxed{5.05 \times 1e31 \, Watt \approx \frac{dK_c}{dt}}
	\]
	
	\section{\color{teal} Question 3 - Accretion disk Maximum temperature}
	
	Given, the Temperature of Acrretion Disk:
	\[
		T_d(r) = T_0 (R/r)^{3/4}(1-\sqrt{\frac{R}{r}})^{1/4}
	\]
	Taking its derivative and setting it to zero:
	\[
		\frac{dT}{dr} = 0 = \frac{3}{4}(\frac{R}{r})^{-1/4}[R\times(-\frac{1}{r^2})][1-\sqrt{\frac{R}{r}}]^{1/4} + (\frac{R}{r})^{3/4}\frac{1}{4}(1-\sqrt{\frac{R}{r}})^{-3/4}[-\sqrt{R}(-\frac{1}{2}r^{-3/2})]
	\]
	\[
		0 = -\frac{6}{r}(1-\sqrt{\frac{R}{r}}) + \sqrt{\frac{R}{r^3}} 
	\]
	\[
		\Rightarrow \boxed{\sqrt{\frac{r_{max}}{R}} = 7/6}
	\]
	Now,
	\[
		T_{max} = T_0 (36/49)^{3/4}[1-\frac{6}{7}]^{1/4} \approx \boxed{0.488 T_0 \approx T_{d-max}}
	\] 
	
	\section{\color{teal} Miller's Method}
	
	Given:
	\[
		(\frac{I_B}{I_P})_{obseved} = (\frac{I_B}{I_P})_{intrinsic} \times 10^{0.4(A_P-A_B)} 
	\]
	Also,
	\[
		A_x = A_V (\frac{\lambda_x}{\lambda_V})^{-\gamma}
	\]
	Substituting, into the above equation:
	\[
		(\frac{I_B}{I_P})_{obseved} = (\frac{I_B}{I_P})_{intrinsic} \times 10^{0.4\times A_V[(\frac{\lambda_P}{\lambda_V})^{-\gamma}- (\frac{\lambda_B}{\lambda_V})^{-\gamma}]}
	\]
	Putting in the given values:
	\[
		(\frac{I_B}{I_P})_{obseved} = 0.93 \times 10^{0.4\times A_V[(\frac{4653}{550})^{-1.3}- (\frac{2166}{550})^{-1.3}]} \approx 0.93 \times 10^{A_V(2.068)}
	\]
	
	For $A_v = (0,5, 100 points)$ the $\frac{I_B}{I_P}_{observed}$ is plot in file 1.
	
	Also for the below lines(nm) with same Upper Energy level (15) the  relative intensities are in the middle column:
	
	line1 :\\
	  91.5823751      5600		15\\
	line2:\\
	  371.19774       3300      15\\
	line 3:\\
	  854.5383       1400      15\\
	  
	\section{\color{teal}Question 5- Pressure, Thermal,  Gravity Balance}
	 \subsection{(a)}
	 Given,
	 \[
	 	4\pi R_{cl}^3 P_0 = 3c_s^2 M_{cl} - \frac{3}{5}\frac{G M_{cl}^2}{R_{cl}}
	 \]
	It is evident that as R$_{cl}$ decreases, for a fixed M$_{cl}$, P$_0$, c$_s$, Gravitational self energy overwhelms kinetic(Thermal) pressure.
	\[
		\frac{3}{5}\frac{GM_{cl}^2}{R_{cl}} + 4\pi R_{cl}^3 P_0 \leq 3c_s^2 M_{cl}
	\]
	\[
		\Rightarrow  4\pi R_{cl}^4 P_0 - 3c_s^2 M_{cl} R_{cl} \leq \frac{3}{5}GM_{cl}^2
	\]
	Is of the form:
	\[
		a R_{cl}^4 - b R_{cl} < c...........(5.1)
	\]
	(for a,b,c $>$ 0)
	
	The equation:
	\[
		ax^4 - bx - c = f(x)
	\]
	\begin{figure}[h]
		\centering\includegraphics[scale=0.51]{Q5_A4.png}
		\caption{Plot of f(x)(https://www.desmos.com/calculator)} 
		\label{fig:figure1}
	\end{figure}
	
	has atleast one root in x$>$0, because at x=0 f(0) $<$ 0 and as x$\rightarrow \infty$ f(x) $\rightarrow \infty$.\\
	Now see that by same logic there is also only one root in $x \epsilon (-\infty, 0)$\\
	Now, lets look at the derivative of the function:
	\[
		f'(x) = 4ax^3 - b
	\]
	has only one real root so the function only has one turning point; so f(x) has only two roots as above.\\
	In our range of interest($R_{cl}>0$) only one root exists (say $R_{cl}*$) so for $0<R_{cl}<R_{cl}*$ f($R_{cl}$) $<0$ as required by inequality(5.1)
	
	\fbox{Hence,  R$_{cl}<$ R$_{cl}*$ for stability}
	
	\subsection{(b)}
	
	From the equation:
	\[
		P_0 = \frac{3c_s^2M_{cl}}{4\pi R_{cl}^3} - \frac{3}{20\pi}\frac{GM_{cl}^2}{R_{cl}^4}
	\]
	is of the form
	\[
		y = \frac{a_1}{x^3} - \frac{b_1}{x^4}
	\]
	(for $a_1,b_1 > 0$ from above)
	Plot of the above function for $x>0$ is as follows:
	\begin{figure}[h]
		\centering\includegraphics[scale=0.51]{Q5b_A4.png}
		\caption{Plot of y =f1(x)(https://www.desmos.com/calculator)} 
		\label{fig:figure1}
	\end{figure}
	
	The peak (of maximum pressure) would exist at x* (equivalently $R_{cl}*$).
	
	For x*($\neq 0$):
	\[
		y'(x)\vline_{x*} = \frac{-3a_1}{x*^4} + \frac{4b_1}{x*^5} = 0
	\]
	\[
		\Rightarrow x* = \frac{4b_1}{3a_1} \Rightarrow R_{cl}* = \frac{4}{3}\frac{3/20\pi GM_{cl}^2}{3c_s^2M_{cl}/4\pi} = \boxed{\frac{4}{15} \frac{GM_{cl}}{c_s^2} = R_{cl}*}
	\]
	The $R_{a-cl}*$  would be the root of the curve and $R_{b-cl}$ would be the maximiser so they are different.
	 
	\subsection{(c) Stability Analysis}
	\begin{figure}[h]
		\centering\includegraphics[scale=0.51]{Q5c_A4.png}
		\caption{Plot of y =f1(x)(https://www.desmos.com/calculator)} 
		\label{fig:figure1}
	\end{figure}
	\textit{Stable Branch(R$> R_{cl}*$):}\\
	
	For any R$> R_{cl}*$, if at that R we increase R a little we see that since P vs R is decreasing the cloud needs P(r+$\delta R$) $<$ P(R) to remain stable. Hence, P(R) decreases the R + $\delta$R perturbation.\\
	Now similarly if we decrease R; we see that P(R-$\delta$R)$>$P(R), and since P(R) is smaller the cloud is free to expand and negates the perturbation.
	
	\textit{Unstable Branch (0$<$R$<R_{cl}*$):}
	
	For any R in this branch if we perturb to R +$\delta$R we see that the P(R + $\delta$R) $>P(R)$ required to stabilise the cloud. But, since P(R) is smaller the cloud freely expands until it reaches the stable branch.\\
	Similarly, for a perturbation P(R- $\delta R$) the pressured required to stabilise the cloud P(R + $\delta R$)$<P(R)$, so the clouds shrinks to zero radius.
	 
	
	
	
	\section{\color{teal} Question 6 : IMFs}
	
	\subsection{Fractions and Numbers}
	
	Given: in mass range M $\rightarrow$ M + dM, dN number of stars occur (0.4 M$_\odot<$M$<$100M$_\odot$)
	
	Hence:
	\[
		N(M\rightarrow 100M_\odot) \propto \int_M^{100M_\odot} M^{-2.35} = \frac{M^{-1.35}}{-1.35} \vline_M^{100M_\odot} = \frac{\frac{1}{M^{1.35}} - \frac{1}{100M_\odot^{1.35}}}{1.35}
	\]
	Now for M  = 8M$_\odot$
	\[
		N(8M_\odot \rightarrow 100M_\odot) \propto \frac{\frac{1}{8^{1.35}} - \frac{1}{100^{1.35}}}{1.35M_\odot^{1.35}} = N_x...{say}	
	\]
	Similarly,
	\[
		N(0.4 M_\odot \rightarrow 100 M_\odot) = 10^{11} \propto \frac{\frac{1}{0.4^{1.35}} - \frac{1}{100^{1.35}}}{1.35M_\odot^{1.35}} 
	\]
	Hence the fraction of stars M $> 8M_\odot$
	\[
		\frac{N_x}{10^{11}} = \frac{\frac{1}{8^{1.35}} - \frac{1}{100^{1.35}}}{\frac{1}{0.4^{1.35}} - \frac{1}{100^{1.35}}} \approx \boxed{0.017 \approx \frac{N_x}{10^{11}}}
	\]
	Hence, N$_x$:
	\[
		\boxed{N(8M_\odot \rightarrow 100M_\odot) \approx 1.7 \times 10^9} 
	\]
	Average mass of stars from N($8M_\odot \rightarrow 100 M_\odot$)
	\[
		\bar{M} = \frac{\int_{8M_\odot}^{100M_\odot}(M )M^{-1.35}}{\int_{8M_\odot}^{100M_\odot} M^{-1.35}} = \frac{\int_{8M_\odot}^{100M_\odot} M^{-0.35}}{\int_{8M_\odot}^{100M_\odot} M^{-1.35}}
	\]
	\[
		\Rightarrow \frac{\frac{M^{0.65} \vline_{8M_\odot}^{100M_\odot}}{0.65}}{\frac{M^{-0.35} \vline_{8M_\odot}^{100M_\odot}}{-0.35}} \approx 0.54 M_\odot \frac{100^{0.65}- 8^{0.65}}{\frac{1}{8^{0.35}}-\frac{1}{100^{0.35}}} \approx \boxed{30.65 M_\odot \approx \bar{M}(8M_\odot \rightarrow 100M_\odot)}
	\]
	
	\subsection{(b) Iron Abundances}
	
	Mass of Iron ejected = \( M_{Fe} = 0.05 M_\odot \times N_x \)\\
	Fraction of mass of this much Iron(based on $\bar{M}$ as in part a):
	\[
		Z = \frac{M_{Fe}N_x}{\bar{M}N_x}  = \frac{0.05 M_\odot}{30.65 M_\odot} =\approx \boxed{0.0016 \approx Z_{Fe}}
	\]
	Since Sun has nearly the same composition of Iron as in ISM, this means Sun was made from ISM after all the supernovae had enreiched it with this amount. All this Iron is sitting dormant in Sun as it majorly just burning Hydrogen. In that sense its a Second generation star(formed from material of its immediate ancestors).
	
	
	
\end{document}

