\documentclass[11pt]{article}

\usepackage{sectsty}
\usepackage{graphicx}
\usepackage{xcolor}
\usepackage{setspace}
\usepackage{hyperref}
\usepackage{gensymb}
\usepackage{amsmath} % for the matrix environments
\usepackage{amsfonts}
\usepackage{amssymb}
\usepackage{cancel}
\usepackage{bbold}
% Margins
\topmargin=-0.90in%-0.45%
\evensidemargin=0in%0in%
\oddsidemargin=0in
\textwidth=7in
\textheight=10.0in
\headsep=0.5in
\onehalfspacing

\title{Astronomy \& Astrophysics Assignment - 3}
\author{\textbf{\Large Swanith Upadhye}}
\date{\today}

\begin{document}
	
	\maketitle
	\noindent\hrulefill
	\Large
	
	\section{\color{teal} Question 1}
	
	The Pressure hydrostatic equilibrium equation:
	\[
		dP = -\rho g dr
	\]
	Multiply and divide by mass absormption coefficient on right side:
	\[
		dP = (-\rho \kappa_\nu dr) \, \frac{g}{\kappa_\nu} 
	\]
	The optical depth \(d\tau_\nu  = \rho \kappa_\nu dr\); Hence:
	\[
		\boxed{\frac{dP}{d\tau_\nu} = - \frac{g}{\kappa_\nu}}
	\]
	
	\section{\color{teal}Question - 2}
	
	\subsection{Pressure vs radius profile}
	
	The Hydrostatic Equilibrium Pressure equation :
	\[
		\frac{dP}{dr} = -\frac{GM(r)\rho}{r^2} = \frac{4G\pi\rho^2}{3}r
	\]
	Since $\rho$ is given a constant,\\
	Now integrating:
	\[
		P(r) = -\frac{2G\pi\rho^2}{3}r^2 + c
	\]
	Given P(R) = 0
	\[
		\Rightarrow c = \frac{2G\pi\rho^2}{3}R^2
	\]
	Hence,
	\[
		\boxed{P(r) = \frac{2G\pi\rho^2}{3}(R^2-r^2)}
	\]
	\subsection{Temperature profile vs r}
	
	Given the star is made up of ideal gas:
	\[
		P = \frac{\rho kT}{\mu m_H} \Rightarrow T = \frac{\mu m_H}{\rho k}P(r) 
	\]
	\[
		\Rightarrow \boxed{T(r) = \frac{2G\pi\rho\mu m_H}{3k}(R^2-r^2)}
	\]
	\subsection{Nuclear Energy Production rate vs r}
	
	Given \(\epsilon \, \alpha \, T^4 \rightarrow (R^2 - r^2)^4\)
	
	\[
		\frac{d\epsilon}{\epsilon} = \frac{4(R^2-r^2)^3 (-2r)}{(r^2-r^2)^4} = \frac{4(-2r)}{R^2-r^2}dr
	\]
	Integrating both sides:
	\[
		\ln(\epsilon/\epsilon_0) = 4\ln(\frac{R^2-r^2}{R^2})
	\]
	where $\epsilon_0$ is Nuclear energy production rate at stellar center.\\
	Given$\epsilon/\epsilon_0 = 0.1$,
	\[
		\frac{1}{4}\ln(0.1) = \ln(1-r^2/R^2) = \ln(0.1^{1/4})
	\]
	Comparing arguments within the log:
	\[
		\Rightarrow r = R\sqrt{1-0.1^{1/4}}
	\]
	Volume fraction for this radius:
	\[
		\boxed{V(r)/V(R) = (1-0.1^{1/4})^{3/2} \approx 0.29 =29 \%}
	\]
	
	\section{\color{teal} Question 3}
	
	\subsection{M vs r}
	
	From the Mass density relation:
	\[
		dM = 4\pi \, \rho(r)\,  r^2 dr
	\]
	For the Given density profile
	\[
		dM =4\pi\rho_c(r^2 - r^3/R)dr
	\]
	Integrating on both sides:
	\[
		\boxed{M(r) = 4\pi\rho_c (\frac{r^3}{3} -\frac{r^4}{4R})}
	\]
	\subsection{Total Mass of the star}
	
	For r=R, in the above relation
	\[
		\boxed{M = \frac{\pi\rho_cR^3}{3}}
	\]
	\subsection{Hydrostatic Equilibrium Pressure Profile}
	
	The Hydrostatic Equilibrium pressure equation:
	\[
		\frac{dP}{dr}  = -\frac{GM(r)\rho_(r)}{r^2}
	\]
	Substituting above M(r) and $\rho(r)$:
	\[
		\frac{dP}{dr} = -4\pi G \rho_c^2 (\frac{r}{3} - \frac{r^2}{4R})(1-\frac{r}{R}) \rightarrow (\frac{r}{3} -\frac{7r^2}{12R} + \frac{r^3}{4R^2})
	\]
	Integrating both sides
	\[
		\cancel{P(R)} + \boxed{P(r) = 4\pi G \rho_c^2 \{\frac{R^2-r^2}{6} - \frac{7}{36R}(R^3-r^3) + \frac{1}{16R^2}(R^4-r^4)\}}
	\]
	
	\section{\color{teal}Question - 4}
	
	\subsection{Mean molecular mass}
	The mean particle mass for Completely ionised $^{12}C$ is: 
	\[
		\mu = \frac{ A_C n_C}{n_C(1+z_C)} = \frac{12\times 1}{ 1+6} = \boxed{12/7 m_H=\bar{m} }	
	\]
	
	\subsection{Scaling relations}
	
	The average Kinetic Energy of stellar components:
	\[
		E_T = \frac{3MKT}{2\bar{m}} 
	\]
	The average Gravitational potential energy of the stellar components:
	\[
		E_G = GM^2/R
	\]
	From Virial theorem:
	\[
		2E_T = E_G \Rightarrow R = \frac{G\bar{m}M}{3KT}
	\]
	Radius relates to the T,M and $\bar{m}$ as:
	\[
		\boxed{R \, \alpha \, \frac{\bar{m}M}{T}}
	\]
	In comparison to Sun's parameters:
	\[
		\frac{R}{R_\odot} = \frac{\bar{m}M/T}{\bar{m_\odot}M_\odot/T_\odot} \approx \frac{1.714 \times 10}{0.62 \times 40} \boxed{\approx 0.691 = R/R_\odot}
	\]
	\subsection{Surface temperature}
	\[
		T = [\frac{L}{4\pi R^2 \sigma}]^{1/4}
	\]
	With respect to the Sun's surface Temperature:
	\[
		\frac{T}{T_\odot} = [\frac{L/L_\odot}{(R/R_\odot)^2}]^{1/4} \approx [10^7/0.691^2]^{1/4} \boxed{\approx 67.64 = \frac{T}{T_\odot}}
	\]
	\subsection{The fraction of mass converted to energy}
	
	The mass defect thats converted to energy = 2$\times$12 - 23.985 = 0.015 au =$\triangle m$\\
	Hence the fraction of mass converted to energy = 0.015/24 = \boxed{0.0625\%}
	
	\subsection{Time taken by star to use up 10\% of its Carbon}
	
	10\% of the mass of the star= M$_{10\%}$ = \(2\times 10^{30}\) kg\\
	is the required mass of the Carbon used.\\
	The Energy of fusion of two $^12C$ atoms = E$_f$ = \( \triangle m c^2 = 0.015 \times 1.66 \times 10^{-24} \, g\, \times 9 \times 10^{20} \, (cm/s)^2 = 2.241 \times 10^{-5} \, ergs \)\\
	Now number of reactions required:
	\[
		N = \frac{1}{2} \frac{M_{10\%}}{A_C \, amu} = \frac{1}{2}\, \frac{2\times10^{30}}{12 \times 1.66\times 10^{-27}} = 5.02 \times 10^{55}
	\]
	where, factor of 1/2 is because two atoms are used up for a single fusion.\\
	Now the Energy released for the 10\% stellar mass:
	\[
		E = N \times E_f = 5.02\times 10^{55} \times 2.241 \times 10^{-5} \,ergs  = 1.125 \times 10^{51} ergs
	\]
	Now the Rate at which the M$_{10\%}$ is used:
	\[
		T = \frac{E}{L} = \frac{E}{10^7 L_\odot} = \frac{1.125 \times 10^{51}}{3.828 \times 10^{40}} \approx 2.94 \times 10^{10} s \approx \boxed{932.27 \, yrs \approx T_{10\%}}
	\]
	
	\section{\color{teal} Question 5 - Convective Transport}
	
	\subsection{Conditition}
	\begin{figure}[h]
		\centering\includegraphics[scale=0.51]{Convection.png}
		\caption{The thermodynamic parameters for and around a blob of stellar matter out of equilibrium (Reference: StellarAstrophysics-Lecture10-notes -Prof. Manoj Purvankara -TIFRM)} 
		\label{fig:figure1}
	\end{figure}
	
	From the above figure, a small parcel of stellar matter experiencing local fluctuations pushes it out of hydrostatic equilibrium.\\
	The gas parcel pushed up undergoes a adiabatic change in volume, since the blobs response is faster than impulse from surrounding(which happen at sound speeds).\\ 
	
	The adiabatic process: PV$^\gamma$ =const $\Rightarrow$ $P \, \propto \rho^\gamma$\\
	
	Taking log and derivative on both sides:
	\[
		\frac{\delta P}{P} = \gamma \frac{\delta\rho}{\rho}.........(5.11)
	\]
	For convection to set in, the density of blob should be smaller than its surrounding:
	\[
		 \delta\rho < d\rho \rightarrow \frac{\delta\rho}{\rho}<\frac{d\rho}{\rho} 
	\]
	for any positive $\rho$.
	Plugging in equation (5.11) in the above relation:
	\[
		\frac{1}{\gamma}\frac{\delta P}{P} < \frac{d\rho}{\rho}............(5.12)
	\]
	Now for the surrounding ideal gas:\(P \propto \rho T\)\\
	Taking similar log derivative on both sides:
	\[
		\frac{dP}{P} = \frac{d\rho}{\rho}+\frac{dT}{T} \Rightarrow \frac{d\rho}{\rho} = \frac{dP}{P} - \frac{dT}{T}
	\]
	Plugging this back into relation (5.12)
	\[
		\frac{1}{\gamma}\frac{\delta P}{P}<\frac{dP}{P} - \frac{dT}{T} \Rightarrow \frac{dT}{T}<\frac{dP}{P}-\frac{1}{\gamma}\frac{\delta P}{P} \Rightarrow \boxed{|\frac{dT}{T}| \gtrsim |\frac{dP}{P}|(1-\frac{1}{\gamma})}
	\]
	In the last line we let \(dP\approx\delta P\) and since \(dT \& dP\) are both negative (dP(r+dr)-dP $<$ 0, same for T) taking the modulus reverses the inequality sign(multiplying by -1 on both sides).\\
	
	Replacing the small quantities dT and dP as respective derivatives wrt r:
	\[
		\frac{dT}{dr} > \frac{T}{P}\frac{dP}{dr}(1-\frac{1}{\gamma})
	\]
	Now on the right hand side substituting the hydrostatic pressure condition:
	\[
		\frac{dT}{dr}> \frac{T}{P}\frac{GM\rho}{r^2}(\frac{1}{\gamma}-1)
	\]
	where the minus sign was absorbed into the $\gamma$bracket.\\
	Now using the ideal gas law:\(P = \frac{\rho k_BT}{\mu m_H}\), we get the required:
	\[
		\boxed{\frac{dT}{dr} > \frac{GM\mu m_H}{k_B r^2}(\frac{1}{\gamma} - 1)}
	\]
	
	\subsection{Schwarzschild Condition}
	
	If the temperature gradient set in by radiative transfer is greater than the minimum possible temperature gradient required for convections, then stable convection ensues:
	\[
		|\frac{dT}{dr}|\, \vline_{\, radiative} > |\frac{dT}{dr}|\, \vline_{\, convective-minimum}
	\]
	Hence, from radiative transfer equation:
	\[
		\frac{3\kappa(r)L(r)\rho(r)}{16\pi r^2acT^3(r)}>\frac{GM\mu m_H}{k_B r^2}(\frac{1}{\gamma} - 1)
	\]
	Hence:
	\[
		L > \{\frac{Gm_H}{k_B}\frac{16\pi4\sigma}{3}(\frac{1}{\gamma}-1)\}\frac{\mu T^3M}{\kappa\rho}
	\]
	
	Putting the values (monoatomic gas $\gamma$ = 5/3, a=4$\sigma$/c above)
	\[
		L > \frac{6.67\times10^{-8} 1.67\times 10^{-24}}{1.38\times 10^{-16}}\frac{64\pi5.67\times10^{-5}}{3}(-2/5)\frac{\mu T^3M}{\kappa\rho}
	\]
	Due to the minus sign the inequality flips:
	\[
		\boxed{L<1.22 \times 10^{-18} \,\, \frac{\mu T^3M}{\kappa\rho}}
	\]
	
	\subsection{Kramers Law Insights}
	
	From The Radiative Temperature gradient relation, we see as $\rho$ increases, the temperature gradient increases as it is in the numerator.\\
	After a critical density the condition\(\frac{dT}{dr}\, \vline_{radiation}>\frac{dT}{dr} \, \vline_{convection}\) will be satisfied and convection starts.\\
	
	Since the opacity increases it becomes more difficult for the radiation to escape and carry the energy with it. To maintain equilibrium after above critical condition convection commences. 
	
	\section{\color{teal} Question 7- Solar Convection Zone}

	In the small convective zone mass approximation M(r) = M$\odot$ is a constant.\\
	From the final boxed result of equation of section (5.1) in this manuscript(convection temperature gradient relation and ideal gas law):
	\[
		\frac{dT}{dr} = \frac{GM_\odot\mu m_H(\frac{1}{\gamma}-1)}{k_B} \frac{1}{r^2} = C_{say} /r^2
	\]
	Hence, intergrating b.s. wrt r:
	\[
		\boxed{T(r) - T_s = C (1/r - 1/R_\odot)}
	\]
	where $T_s$ stands for surface temperature.\\
	From the Hydrostatic pressure equilibrium relation:
	\[
		\frac{dP}{dr} = \frac{-GM_\odot \rho(r)}{r^2} = -\frac{G M_\odot}{r^2} \frac{P\mu m_H}{RT}
	\]
	\[
		\Rightarrow \frac{dP}{P} = - \frac{G\mu m_H}{R}\frac{1}{r^2}\frac{1}{T_s + C(1/r - 1/R_\odot)}dr  =-D1_{say}\frac{1}{(T_s-1/R_\odot)r^2 +Cr}dr 
	\]
	\[
		= -D_1\frac{dr}{D2_{say}r^2+Cr} = -\frac{D_1}{D_2}\frac{dr}{(r + C/2D_2)^2-C^2/4D_2^2}
	\]
	Integrating both sides:
	\[
		\boxed{ln(P/P_s) = \frac{-D_1}{C}\ln|\frac{r(R_\odot+C/D_2)}{(r+C/D_2)R_\odot}|}
	\]
	where we used:
	\(
	 	\int \frac{1}{x^2-a^2} = \frac{1}{2a}ln|\frac{x-a}{x+a}|
	\)\\
	Now the density, from ideal gas law:
	\[
		\rho(r)  = \frac{P(r)\mu m_H}{RT(r)} = \boxed{P(s)exp(-\frac{-D_1}{C}\ln|\frac{r(R_\odot+C/D_2)}{(r+C/D_2)R_\odot}|)\frac{\mu m_H}{R[T_s + C(1/r-1/R_\odot)]}=\rho(r)}
	\]
	
	\section{\color{teal}Question 8 - Gamow integral}
	
	\subsection{Extrema of Transition Probability factor}
	
	Given:
	\[
		f(E) = \exp(-E/k_BT)\exp(-\sqrt{E_G/E})
	\]
	Differentiating with respect to E:
	\[
		f'(E)\, \vline_{\, E_0 = 0} = \exp(-E/k_BT)\exp(-\sqrt{E_G/E})[-\frac{1}{k_BT} +\frac{1}{2}\frac{\sqrt{E_G}}{E^{3/2}}] \, \vline_{E_0=0}
	\]
	\[
		\boxed{E_0 = (\frac{k_BT}{2})^{2/3}E_G^{1/3}}
	\]
	\subsection{Finding the width in Gaussian approximation}
	
	Upon Taylor expanding the exponent in \(\exp(-(E/k_BT + \sqrt{E_G/E}))\)
	\[
		f(E) = -(E/k_BT + \sqrt{E_G/E}) = -f(E_0) - (E-E_0)\cancel{f'(E_0)} - \frac{1}{2}(E-E_0)^2 f''(E_0) + O((E-E_0)^3)
	\]
	Comparing to a Gaussian: its exponent : \(\frac{-1}{2\triangle^2}(E-E_0)^2\)
	\[
		\triangle^2 = \frac{1}{f''(E_0)}
	\]
	Now f''(E$_0$):
	\[
		f''(E) \, \vline_{E_0} = [0 + \frac{3}{4}\sqrt{E_G}E^{-5/2}]\, \vline{\, E_0}
	\]
	\[
		 = \frac{3}{4}E_G^{1/2} (\frac{k_BT}{2})^{2/3 \times -5/2} E_G^{1/3 \times -5/2} = \frac{-3}{4}E_G^{-1/3}(\frac{k_BT}{2})^{-5/3}
	\]
	\[
		\Rightarrow \triangle = \sqrt{1/f''(E_0)} =\{\frac{3}{4}E_G^{-1/3}(\frac{k_BT}{2})^{-5/3}\}^{-1/2} = \frac{3^{-1/2}}{2^{(-2+5/3)/2}}E_G^{1/6}(k_BT)^{5/6}
	\]
	\[
		= \boxed{\triangle = \frac{2^{1/6}}{3^{1/2}}E_G^{1/6}(k_BT)^{5/6}}
	\]
	
	\subsection{Gaussian Approximation Area}
	
	Area:
	\[
		\int_{0}^{\infty} f(E_0) exp(-\frac{1}{2\triangle^2}(E-E_0)^2) dE = f(E_0)\frac{\sqrt{\pi}}{\sqrt{\frac{1}{2\triangle^2}}} = \boxed{f(E_0)\sqrt{2\pi}\triangle = Area}
	\]
	since E$_0$>0 limits$\infty\rightarrow\infty \approxeq 0\rightarrow \infty$ 
	
	\section{\color{teal} Question 9 - Solar Wind Mass Loss}

	The Continuity equation for mass loss:
	\[
		\frac{d\rho}{dt} + \nabla.\bar{J} = 0
	\]
	The mass current:
	\[
		\bar{J} = \rho \bar{v} 
	\]
	Integrating the continuity equation wrt volume:
	\[
		\int \frac{d\rho}{dt}dV + \int \nabla \rho \bar{v} dV
	\]
	Now  \(\rho dV = dM \) and applying Gauss Law in the second term :
	\[
		\frac{\partial}{\partial t} \int dM + \oint_{\partial V} \rho \bar{v}.\hat{n} dS 
	\]
	Since on an average all the integrands are constant
	\[
		-\frac{\partial}{\partial t} \triangle M = \rho \bar{v}.\hat{n} S
	\]
	where S is the area of sphere, with radius as big as Earth-Sun distance 1 AU\\
	Integrating to the current age of sun T:
	\[
		-\triangle M(T)+\cancel{\triangle M(0)} = \rho \bar{v}.\hat{n} S T
	\]
	Hence:
	\[
		|\triangle M(T)| = (10 \times 1.66 \times 20^{-24} g/cm^3) \times (400 \times 10^5 cm/s) \times (4\pi \{150 \times 10^6 10^5 cm\}^2)
	\]
	\[
		 \times \{4.5 \times 10^9 \times 365 \times 24 \times 3600 s\}
	\]
	where we used on an average $\bar{v}$ is a constant so $\bar{v}.\hat{n} = |\bar{v}| = 400 km/s$ given.\\
	
	\[
		|\triangle M(T)| \approx 2.66 \times 10^{29} g
	\]
	For solar mass 1.99 $\times$ 10$^{33}$ g:
	\[
		\boxed{\triangle M(T) \approx 0.0134 \%}	
	\]
	
	\section{\color{teal} Question 10 - Scaling Relations}
	
	\subsection{Main sequence lifetimes}
	
	The Mass-Luminosity relation for:
	\begin{itemize}
		\item Low mass star: L $\propto M^5 $
		\item Intermediate star(like Sun): L $\propto M^3$
		\item Massive Star: L $\propto$ M 
	\end{itemize}
	
	The main sequence life time $T_ms$ is directly proportion al to the mass of the star(amount of nuclear fuel) and inversely related to Luminosity(rate of that energy production) 
	
	The corresponding lifetimes:
	
	\begin{itemize}
		\item Low mass star :$T_ms \propto M/M^5 \rightarrow M^{-4}$
		\item Intermediate Mass Star : $T_ms \propto M/M^3 \rightarrow M^{-2}$
		\item Heavy Star : $\boxed{T_ms \propto M/M \rightarrow M^0}$.....(Q10 - part(iii))
	\end{itemize}
	
	Now for the question:
	\begin{itemize}
		\item Low Mass M = 0.5 M$_\odot$\\ : $\boxed{T_ms = T_\odot (M/M_\odot)^{-4} = T_\odot (0.5)^{-4} = 1.6 \times 10^{11} years > t_{universe}}$\\
		$t_{universe}  = $13.8 billion years $< T_ms$ above = 160 billion years)
		
		\item Massive Star (M = 10 M$\odot$ T): Can not be estimated in this way since $T_{ms} \propto M^0$ but is expected to be less than T$_\odot$
		 
	\end{itemize}
	
	\subsection{Sun swells}
	
	Using Stefan Boltzmann Law:
	\[
		L \propto R^2 T^4
	\]
	Given R = 200 $R_\odot$, T = $\frac{1}{2}T_\odot$
	\[
		L = L_\odot 200^2 \frac{1}{2^4} = 2500 L_\odot
	\]
	
	
	
\end{document}