\documentclass[11pt]{article}

\usepackage{sectsty}
\usepackage{graphicx}
\usepackage{xcolor}
\usepackage{setspace}
\usepackage{hyperref}
\usepackage{gensymb}
\usepackage{amsmath} % for the matrix environments
\usepackage{amsfonts}
\usepackage{amssymb}
\usepackage{cancel}
\usepackage{bbold}
% Margins
\topmargin=-0.90in%-0.45%
\evensidemargin=0in%0in%
\oddsidemargin=0in
\textwidth=7in
\textheight=10.0in
\headsep=0.5in
\onehalfspacing

\title{QFT Assignment - 1}
\author{\textbf{\Large Swanith Upadhye}}
\date{\today}

\begin{document}
	
	\maketitle
	\Large
	
	\section*{\color{teal} Question 1 - Klein Paradox}
	
	\subsection{The Dirac equation} 
	\[
		(i\gamma^\mu \partial_\mu - m)\psi = 0\\
		\Rightarrow (-i\gamma^0\partial_0 + i\gamma^k \partial_k - m)\psi =0
	\]
	\(\partial_0 = -\partial_t\) and \(\partial_k = \nabla\);substituting above
	\[
		i\gamma^0\partial_t\psi = -i\gamma^k\nabla\psi +m\psi
	\]
	Multiplying by \(\gamma_0\) on both sides to the left.
	\[
		(\gamma^0)^2 H\psi = \gamma^0\gamma^k p_k\psi + m\psi
	\]
	\((\gamma^0)^2=1\) and \(\gamma^0\gamma^k = \alpha^k\); Hence
	\[
		H\psi = \alpha^k p_k\psi + m\gamma^0\psi
	\]which is as required.
	In the above:-
	\[
		\gamma^0 = \beta = \begin{bmatrix}\mathbb{1} & 0 \\0 &-\mathbb{1} \end{bmatrix}\,
		 \gamma^k = \begin{bmatrix}0 & \mathbb{\sigma}^k \\-\mathbb{\sigma}^k & 0 \end{bmatrix}
	\]
	
	\subsection{Particle at rest}
	For a particle at rest \(p_k = 0\); Hence H = \(m\gamma^0\)
	Hence its eigenvalues are \({m,m,-m,-m}\) and the respective eigenvectors are:
	\[
		w_1(0)=\begin{bmatrix} 1\\0\\0\\0\end{bmatrix} \, w_2(0)=\begin{bmatrix} 0\\1\\0\\0\end{bmatrix} \, w_3(0)=\begin{bmatrix} 0\\0\\1\\0\end{bmatrix} \, w_4(0)=\begin{bmatrix} 0\\0\\0\\1\end{bmatrix}
	\]
	Hence the wavefunction :\\
	 \(\psi_{1,2}(x)= w_{1,2}(0)exp(-imc^2t/\bar{h})\)  and  \(\psi_{3,4}(x)= w_{3,4}(0)exp(+imc^2t/\bar{h})\)
	 
	 The energy of \(\psi_{3,4}(x)\) is -mc$^2<0$; as suggested by the eigenvalues of \(w_{3,4}(0)\) 
	 
	 \subsection{Particle of mass m and momentum $p_k$} 
	 
	 The Hamiltonian for this case becomes:
	 \[
	 	H = \alpha^k p_k +m\gamma^0 = 
	 	\begin{bmatrix}0 & \sigma^k\\\sigma^k & 0 \end{bmatrix}p_k + m	\begin{bmatrix}\mathbb{1} & 0\\0 & -\mathbb{1} \end{bmatrix}
	 \]
	 Expanding the matrices:
	 \[
	 	H = \begin{bmatrix}m&0&p_z&p_x-ip_y\\0&m&p_x+ip_y&-pz\\pz&p_x-ip_y&-m&0\\p_X+ip_y&-p_z&0&-m\end{bmatrix}
	 \]
	 The eigenvalues for this equation from mathematica are\\
	 \[
	 	\{\sqrt{m^2+p_x^2+p_y^2+p_z^2},\sqrt{m^2+p_x^2+p_y^2+p_z^2},-\sqrt{m^2+p_x^2+p_y^2+p_z^2},-\sqrt{m^2+p_x^2+p_y^2+p_z^2}\}
	 \]
	 The respective eigen vectors are:
	 \[
	 	w_1(0)=\frac{1}{E+m}\begin{bmatrix} E+m\\0\\p_z\\p_x+ip_y\end{bmatrix} \, w_2(0)=\frac{1}{E+m}\begin{bmatrix} 0\\E+m\\p_x-ip_y\\-p_z \end{bmatrix} \, 
	 	w_3(0)=\frac{1}{E+m}\begin{bmatrix} -p_z\\-(p_x+ip_y)\\E+m\\0\end{bmatrix} \, 
	 \]
	 \[
	 	w_4(0)=\frac{1}{E+m}\begin{bmatrix} -(p_x-ip_y)\\p_z\\0\\E+m\end{bmatrix}	
	 \]
	 For the normalisation factor, check
	 \[
	 	w_1(0)^\dagger w_1(0) = \begin{bmatrix}1&0&\frac{p_z}{E+m}&\frac{p_x-ip_y}{E+m}\end{bmatrix}.\begin{bmatrix}1\\0\\\frac{p_z}{E+m}\\\frac{p_x+ip_y}{E+m}\end{bmatrix} = 1 + \frac{p_x^2 + p_y^2 + p_z^2}{(E+m)^2}  = 1 + \frac{(E+m)(E-m)}{(E+m)^2}
	 \]
	 \[
	 	= 1+\frac{E-m}{E+m} = \frac{2E}{E+m}
	 \]
	 Hence the wavefunctions become:
	 \[
	 	\psi_r(x) = \sqrt{\frac{E+m}{2E}}exp(-ip_\mu x^\mu\epsilon_r)w_r(0)
	 \]
	 where \(\epsilon_r = \{+1,+1,-1,-1\}\) for r = \{1,2,3,4\}
	 
	 In the rest frame the energy eigenvalues become:
	 \{m,m,-m,-m\} so \(\psi_{3,4}\) are discarded(E$<0$ solutions)
	 
	\subsection{Potential Step}
	
	 For $z>0$ the Dirac equation becomes
	 \[
	 	\{\begin{bmatrix}\mathbb{E}&0\\0&\mathbb{E}\end{bmatrix}\}\psi =\{ \alpha^k p_k + m \gamma^0 +  \begin{bmatrix}\mathbb{V_0}&0\\0&\mathbb{V_0}\end{bmatrix}\}\psi
	 \]
	 Hence,
	 \[
	 	\{\begin{bmatrix}\mathbb{E-V_0}&0\\0&\mathbb{E-V_0}\end{bmatrix}\}\psi =\{ \alpha^k p_k + m \gamma^0\}\psi
	 \]
	 Hence the solutions are same as above, only replace E$\rightarrow$ E-V$_0$
	 
	 \subsection{Wave function matching}
	 
	 The wave functions(modulo normalisation factors):
	 \[
	 	\psi_i = a\, exp(ik_1z)\frac{1}{E+m}\begin{bmatrix}E+m\\0\\p\\0\end{bmatrix}
	 \]	
	 \[
	 	\psi_r = b\, exp(-ik_1z)\frac{1}{E+m}\begin{bmatrix}E+m\\0\\-p\\0\end{bmatrix} + b'\, exp(-ik_1z)\frac{1}{E+m}\begin{bmatrix}0\\E+m\\0\\-p\end{bmatrix}
	 \]
	 \[
	 \psi_t = d\, exp(ik_2z)\frac{1}{E-V_0+m}\begin{bmatrix}E-V_0+m\\0\\p'\\0\end{bmatrix} + d'\, exp(ik_2z)\frac{1}{E-V_0+m}\begin{bmatrix}0\\E-V_0+m\\0\\p'\end{bmatrix}
	 \]
	 Equating \(\psi_i(z=0)+\psi_r(z=0) = \psi_t(z=0)\), gives
	\[
		\begin{bmatrix}(a+b)\\b'\\(a-b)\frac{p}{E+m}\\-b'\frac{p}{E+m}\end{bmatrix} = \begin{bmatrix}d\\d'\\d\frac{p'}{E-V_0+m}\\d'\frac{p'}{E-V_0+m}\end{bmatrix}
	\]	 	 
	 From the above clearly b'=d'=0;also\\
	%Now equating the derivatives of the wavefunctions wrt z at z=0, gives;
	%\[
	%	\begin{bmatrix}(a-b)(ik_1)\\0\\(a+b)(ik_1)\frac{p}{E+m}\\0\end{bmatrix} = \begin{bmatrix}d(ik_2)\\0\\d(ik_2)\frac{p'}{E-V_0+m}\\0\end{bmatrix}
	%\]
	\[ a+b=d \, and \, a-b = d\frac{k_2}{k_1}\frac{E+m}{E-V_0+m} =rd \]
	where k$_1$=p and k$_2$ = p'. So
	\[
		k_1^2 = E^2 - m^2 
	\]
	\[
		k_2^2 = (E-V_0)^2 - m^2 = (E-V_0-m)(E-V_0+m)
	\]
	\subsection{Transmittance and Reflectance}
	
	The probability currents(modulo normalisation factors):
	\[
		j_i = \psi_i^\dagger \gamma^0\gamma^3 \psi_i = \begin{bmatrix}1&0&\frac{p}{E+m}&0\end{bmatrix} \begin{bmatrix}0&0&1&0\\0&0&0&-1\\1&0&0&0\\0&-1&0&0\end{bmatrix} \begin{bmatrix}1\\0\\\frac{p}{E+m}\\0\end{bmatrix} = \frac{2p}{E+m}|a|^2
	\]
	Similarly,
	\[
		j_r = \psi_r^\dagger \gamma^0\gamma^3 \psi_r = \begin{bmatrix}1&0&\frac{-p}{E+m}&0\end{bmatrix} \begin{bmatrix}0&0&1&0\\0&0&0&-1\\1&0&0&0\\0&-1&0&0\end{bmatrix} \begin{bmatrix}1\\0\\\frac{p}{E+m}\\0\end{bmatrix} = \frac{-2p}{E+m}|b|^2
	\]
	and,
	\[
		j_t = \psi_t^\dagger \gamma^0\gamma^3 \psi_t = \begin{bmatrix}1&0&\frac{p'}{E-V_0+m}&0\end{bmatrix} \begin{bmatrix}0&0&1&0\\0&0&0&-1\\1&0&0&0\\0&-1&0&0\end{bmatrix} \begin{bmatrix}1\\0\\\frac{p'}{E-V_0+m}\\0\end{bmatrix} = \frac{2p'}{E-V_0+m}|d|^2
	\]
	Hence Reflectance:
	\[
		R = |\frac{b}{a}|^2
	\]
	From a+b=d and a-b =rd;
	\[
		(a + b)r = a-b \Rightarrow a(r-1) = b(-1-r) \Rightarrow \frac{a}{b} = \frac{1+r}{1-r}
	\]
	Now,
	\[
		R = \frac{(1-r)^2}{(1+r)^2} \, Hence \, \, , T = 1-R  = \frac{4r}{(1+r)^2}
	\]
	When \(V_0 > E + m\) check that \(r = \frac{k_2}{k_1}\frac{E+m}{E-V_0+m} < 0 \) since \(k_2/k_1\) doesn't change sign in this condition.\\
	Hence, r = - $|r|$ ;
	So \( R = \frac{(1+|r|)^2}{(1-|r|)^2}>1 \dotfill\)(for all $|r|>0$)\\
	There may be particle anti-particle creation for this condition which is increasing the reflected flux.
	
	
	\section{\color{teal} Momentum operator for a Scalar field}
	
	\subsection{Infinitesimal translation}
	Given
	\[
		T(a)\phi(x)T(a)^{-1} = \phi(x-a)
	\]
	For infinitesimal translation \(T(a) = 1 - ia_\mu P^
	\mu\) and \( T(a)^{-1} = 1 + i a_\mu P^\mu\)\\
	Substituting back above:\\
	\[
		\phi(x-a) = (1 - ia_\mu P^
		\mu)\phi(x)(1+ia_\mu P^\mu) \approxeq \phi(x) +ia_\mu(\phi(x)P^\mu - P^\mu \phi(x)) + O(a^2)
	\]
	\[
		\Rightarrow \phi(x) - \phi(x-a) = -ia_\mu[\phi(x), P^\mu]
	\]
	\[
		\Rightarrow i \frac{\phi(x)-\phi(x-a)}{x-(x-a_\mu)}  = [\phi(x),P^\mu]
	\]
	\[
		\Rightarrow \boxed{i \partial^\mu \phi(x) = [\phi(x), P^\mu]}
	\]
	
	\subsection{Time component}
	
	Since \(\partial^0 = \partial^t\) and \(P^0 = H\)\\
	\( \boxed{i\partial ^t \phi= [\phi,H]}\)
	
	\subsection{Klein Gordon Equation using H}
	
	For a free scalar field
	\[
		H = \int \mathcal{H} d^3x = \int d^x [\frac{1}{2}\Pi(x)^2 + \frac{1}{2}\nabla\phi(x)^2 + \frac{1}{2}m^2\phi(x)^2 - \Omega_0]
	\]
	%Hence substituting in Heisenber equation of motion($[\phi(y),H(x)]$only \(\Pi\) term remains):
	%\[
		%\int d^3x \frac{1}{2}[\phi(y), \Pi(x)^2] \rightarrow \Pi(x)[\phi(y),\Pi(x)] + [\phi(y),\Pi(x)]\Pi(x)
	%\]
	%\[
	%	 = \int d^3x \, i \, \delta(x-y)\Pi(x) = i\Pi(y) = i\dot{\phi}(y)
	%\]
	See that,
	\[
		i[\dot{\phi},H] = - (\partial^t)^2 \phi(x)................. (1)	
	\]
	Now for the LHS
	\[
		i\int d^3x[\Pi(y),\frac{1}{2}(\Pi(x)^2  + (\nabla\phi(x))^2 + m^2\phi(x)^2 - \Omega_0)]
	\]
	Only terms that remain,
	\[
		\frac{i}{2} \int d^3x \, \, \nabla \phi(x)[\Pi(y),\nabla\phi(x)] + [\Pi(y),\nabla\phi(x)]\nabla \phi(x) -i 2m^2\delta(x-y)\phi(x) 
	\]
	$\nabla$may be pulled out of the commutator, since it only acts on x terms and not y.
	\[
		\int d^3x \nabla\phi(x)\nabla\delta(x-y) + m^2\phi(x)
	\]	
	Integrating the first term by parts
	\[
		-\int d^3x \nabla^2\phi(x)\delta(x-y) + m^2\phi(x)
	\]
	Equating with (1):
	\[
		0 = ((\partial^t)^2-\nabla^2 + m^2) \phi = \boxed{(-\square +m^2)\phi=0}
	\]
	which is the required KG equation.
	
	\subsection{Free field momentum operator}
	
	\[
		[P^\mu(x),\phi(y)] = \int d^3x [\Pi(x)\nabla\phi(x), \phi(y)] \rightarrow [\Pi(x),\phi(y)]\nabla\phi(x) \rightarrow -i\delta(x-y)\nabla\phi(x)
	\]
	\[
		\Rightarrow \boxed{[P^\mu(x),\phi(y)] = -i \nabla \phi(y)}
	\]
	which is as required by part (1).
	
	\subsection{P in terms of a(k) and a(k)$^\dagger$}
	
	\[
		P = -\int d^3x \dot{\phi}(x)\nabla\phi(x)
	\]
	Using,
	\[
		\phi(x)= \int \tilde{dk} \, a(k) exp(-i\omega t + i\bar{k}.\bar{x})) + a(k)^\dagger exp(i\omega t - i\bar{k}\bar{x})
	\]
	So,
	\[
		\dot{\phi(x)} = \int \tilde{dk} [-i\omega a(k)exp(-i\omega t + i\bar{k}.\bar{x})) + i\omega a(k)^\dagger exp(i\omega t - i\bar{k}\bar{x})]
	\]
	And, 
	\[
		\nabla\phi(x) = \int \tilde{dk} [i\bar{k} a(k) exp(-i\omega t + i \bar{k}.\bar{x}) -i\bar{k} a(k)^\dagger exp(i\omega t - i \bar{k}. \bar{x})]
	\]
	Substituting back into the P equation:
	\[
		P = -\int d^3x \tilde{dk}\tilde{dk'}[-\omega a(k) exp(-i\omega t + i\bar{k}.\bar{x}) + \omega a(k)^\dagger exp(i\omega t - i \bar{k}.\bar{x})] 
	\]
	\[
	 \times [\bar{k'}a(k')exp(-i\omega t + i\bar{k'}.\bar{x}) - \bar{k'}a(k')^\dagger exp(i\omega t - i\bar{k'}\bar{x})]
	\]
	\[
		 = \int d^3x \tilde{dk}\tilde{dk'} \omega \bar{k'}[a(k)exp(ikx) - a(k)^\dagger exp(-ikx)][-a(k')exp(ik'x)+a(k')^\dagger exp(-ik'x)]
	\]
	\[
		= \int d^3x \tilde{dk}\tilde{dk'} \omega \bar{k'} [-a(k)a(k')exp(i(k+k')x) + a(k)a(k')^\dagger exp(i(k-k')x)
	\]
	\[
		+ a(k)^\dagger a(k')exp(-i(k-k')x) - a(k)^\dagger a(k')^\dagger exp(-i(k+k')x)]
	\]
	\[
		=\int \tilde{dk}\tilde{dk'}\omega \bar{k'}(2\pi)^3[-a(k)a(k')\delta(k+k')e^{-i2\omega t} +  a(k)a(k')^\dagger \delta(k-k') + a(k)^\dagger a(k')\delta(k-k')
	\]
	\[
		- a(k)^\dagger a(k')^\dagger\delta(k+k')e^{i2\omega t} ]
	\]
	The integration variable:
	\[
		= (2\pi)^3\int \tilde{dk}\omega \frac{d^3k'}{(2\pi)^3 2\omega} \bar{k'}...........
	\]
	Continuing for P
	\[
		= \int \tilde{dk} \frac{\bar{k}}{2}[a(k)a(k)^\dagger +a(k)^\dagger a(k) - a(k)a(-k)e^{-i2\omega t} +a(k)^\dagger a(-k) e^{i2\omega t} ] 
	\]
	The third and fourth terms are odd and vanish upon the \(d^3\bar{k}\) integral:
	\[
		\Rightarrow P = \int \tilde{dk} \frac{\bar{k}}{2}[a(k)a(k)^\dagger +a(k)^\dagger a(k)]	
	\]
	Now, from $[a(k), a(k')^\dagger] = \delta(k-k')$:
	\[
		= \int \tilde{dk} \frac{\bar{k}}{2}[2a(k)^\dagger a(k) +\cancel{\delta(k-k)}]
	\]
	Finally,
	\[
		\Rightarrow \boxed{P = \int \tilde{dk} \, \bar{k} \, a^\dagger (k) a(k)}
	\]
	
	\section*{\color{teal}Question 3 - Particle Creation by Classical Source}

	\subsection{Equation of Motion for classical source Lagrangian}
	
	Given,
	\[
		\mathcal{L} = -\frac{1}{2}\partial_\mu \phi \partial^\mu \phi - \frac{1}{2} m^2 \phi^2 + j(x)\phi(x)
	\]
	The E-L equation of motion, but first
	\[
		\partial \mathcal{L}/\partial (\partial_\alpha \phi) = - \frac{1}{2} (\delta_{\alpha \mu}\partial^\mu \phi \rightarrow \partial_\alpha \phi + g^{\mu \nu}\delta_{\nu\alpha} \partial_\mu \phi \rightarrow g^{\mu \alpha } \partial_\mu \phi = \partial_\alpha \phi)
	\]
	\[
		= -\partial_\alpha \phi
	\]
	Thus,
	\[
		\partial^\alpha( \partial \mathcal{L}/\partial (\partial_\alpha \phi)) =  -\partial^\alpha \partial_\alpha \phi
	\]
	And,
	\[
		\partial \mathcal{L}/\partial \phi = -m^2 \phi  + j 
	\]
	Finally, equating the above two parts,
	\[
		(-\square + m^2)\phi(x) = j(x)
	\]
	Greens function:
	\[
		(-\square + m^2)D_R(x-y) = -i\delta^4(x-y)
	\]
	Writing the greens function as the Inverse fourier transform of $\tilde{D_p}$
	\[
		D_R(x-y) = \int \frac{d^4p}{(2\pi)^2} \, \tilde{D(p)} \, exp(ip(x-y))
	\]
	
	\subsection{Evaluating the Greens Function}
	
	Substituting into the above equation, LHS
	\[
		\int \frac{d^4p}{(2\pi)^2} \, (p^2 + m^2) \, \tilde{D(p)} \, exp(ip(x-y))
	\]
	And the RHS,
	\[
		-i\int \frac{d^4p}{(2\pi)^2} exp(ip(x-y))
	\]
	Comparing the integrands,
	\[
		\tilde{D(p)} = \frac{-i}{p^2 + m^2}
	\]
	
\end{document}