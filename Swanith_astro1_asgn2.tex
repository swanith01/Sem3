\documentclass[11pt]{article}

\usepackage{sectsty}
\usepackage{graphicx}
\usepackage{xcolor}
\usepackage{setspace}
\usepackage{hyperref}
\usepackage{gensymb}
\usepackage{amsmath} % for the matrix environments
\usepackage{amsfonts}
\usepackage{amssymb}
\usepackage{cancel}
\usepackage{bbold}
% Margins
\topmargin=-0.90in%-0.45%
\evensidemargin=0in%0in%
\oddsidemargin=0in
\textwidth=7in
\textheight=10.0in
\headsep=0.5in
\onehalfspacing

\title{Atronomy \& Astrophysics I -  Assignment - 2}
\author{\textbf{\Large Swanith Upadhye}}
\date{\today}

\begin{document}
	
	\maketitle
	\noindent\hrulefill
	\Large
	
	\section*{\color{teal} Question 1 - Celestial coordinates and Precession}
	
	\subsection{Angular separation between two objects given Celestial coordinates}
	
	Coordinates of Proxima Centauri:
	\[
	(\alpha_P , \delta_P) = (14^h 29^m 42.95^s, -62\degree 40' 46.1'')
	\]
	Converting it all to degrees,
	\[
	\alpha_P = (14 + 29/60 + 42.96/3600) \times 15\degree \approx 217.42896 \degree
	\]
	\[
	\delta_P = -(62 + 40/60 + 46.1/3600) \approx -62.6795\degree
	\]
	
	Similarly for Alpha Centauri $(\alpha_A, \delta_A) = (14^h 39^m 36.50^s , -60\degree50'02.3'')$
	\[
	\alpha_A = (14 + 39/60 + 36.50/3600) \times 15 \degree \approx 219.90208 \degree
	\]
	\[
	\delta_D = -(60+50/60+2.3/3600) \approx -60.83397 \degree
	\]
	
	The unit-vector on the (unit) sphere has the form $(\cos(\delta).\cos(\alpha),\cos(\delta).\sin(\alpha), \sin(\delta))$ where the $\alpha \, and \, \delta $ are celestial coordinates.\\
	
	Hence the coordinates of Proxima Centauri:(cos($\delta_P$) $\approx$ 0.45897, sin($\delta_P$) $\approx$ -0.88845, cos($\alpha_P$) $\approx$ -0.7941, sin($\alpha_P$) $\approx$ -0.6078)
	\[
	P(x,y,z) \approx (-0.3644, -0.2790, -0.8884)
	\]
	Similarly for Alpha Centauri: (cos($\delta_A$) $\approx$ 0.4873, sin($\delta_A$) $\approx$ -0.8732, cos($\alpha_A$) $\approx$ -0.7671, sin($\alpha_A$) $\approx$ -0.6415)
	\[
	A(x,y,z) \approx (-0.3738, -0.3126, -0.8732)
	\]
	Now the dot product of these vectors gives the cosine of the angular separation between these quantities.
	\[
	cos(\phi_{PA}) = P.A =  (-0.3640) \times (-0.3738) + (-0.2786) \times (-0.3126) + (-0.8887) \times (-0.8732) \approx 0.9992
	\]
	\[
	\boxed{\phi_{PA} = \arccos(0.9992) \approx 2.2920\degree \, or \, 0.04 \,rad}
	\]
	
	\subsection{Distance between the objects}
	
	Since Alpha Centauri and Proxima Centauri are gravitationally bound, hence they can't be optical doubles. Hence the distance between them for small angular separation can be approximated by r$\times \theta$ 
	\[
	d_{AP} = r \times \theta = 1.3 pc \times 0.04 \approx 0.052 pc   		
	\]
	
	\subsection{Precession effects on celestial coordinates}
	Reference : \url{https://sceweb.uhcl.edu/helm/WEB-Positional%20Astronomy/Tutorial/Precession/Precession.html}
		
		Since the LuniSolar precession causes the revolution of our celestial axis around ecliptic axis; in our frame the ecliptic will seem to rotate along its axis. This causes the slow shift in position of vernal equinox along the ecliptic.
		
		So lets find the coordinates in ecliptic and celestial coordinates.
		
		\begin{figure}[h]
			\centering\includegraphics[scale=0.6]{Q1c_astro1.png}
			\caption{Celestial Coordinates $(\alpha, \delta)$ and Ecliptic Coordinates$(\lambda, \beta)$ for X ; P and K are the respective poles; $\epsilon$ is the tilt of Earth's Rotational axis} 
			\label{fig:figure1}
		\end{figure}
		
		Using spherical trignometric cosine law in $\triangle$ KPX, for the side PX:
		\[
		cos(PX) = cos(KX)cos(PK) + sin(KX)sin(PK)cos(K)
		\]
		\[
		\rightarrow cos(90 -\delta) = cos(90-\beta)cos(\epsilon) + sin(90-\beta)sin(\epsilon)cos(90-\lambda) 
		\]
		\[
		\Rightarrow sin(\delta) = sin(\beta)cos(\epsilon) + cos(\beta)sin(\epsilon)sin(\lambda).......(1)
		\]
		
		For side KX:
		\[
		cos(KX) = cos(KP)cos(PX) + sin(KP)sin(PX)cos(90 + \alpha)
		\]
		\[
		cos(90 - \beta) = cos(90 - \delta)cos(\epsilon) + sin(90 - \delta)sin(\epsilon)cos(90 + \alpha)
		\]
		\[
		\Rightarrow sin(\beta) = sin(\delta)cos(\epsilon) - cos(\delta)sin(\epsilon)sin(\alpha)..........(2)
		\]
		Now apply sine rule at vertices K and P, to get;
		\[
		sin(90-\delta)/sin(90-\lambda) = sin(90-\beta)/sin(90+\alpha)
		\]
		\[
		\Rightarrow cos(\delta)/cos(\lambda) = cos(\beta)/cos(\alpha)			
		\]
		\[
		\Rightarrow cos(\lambda)cos(\beta) = cos(\alpha)cos(\delta)...................(3)
		\]
		Now lets see how the variation in ecliptic and celestial coordinates are related. Note that since the motion of ecliptic is purely in its plane the ecliptic latitude $\beta$ does not change. Nor does $\epsilon$.
		
		Derivative of equation (1):
		
		\[
		cos(\delta)d\delta/dt = cos(\beta)sin(\epsilon)cos(\lambda)d\lambda/dt
		\]
		Use equation (3) to eliminate $\beta, \lambda, \delta$
		\[
		d\delta/dt = cos(\alpha)sin(\epsilon)d\lambda/dt...........(4)
		\]
		From(2)
		\[
		cos(\beta)\cancel{d\beta/dt} = 0 = cos(\epsilon)cos(\delta)d\delta/dt + sin(\delta)d\delta/dt sin(\epsilon) sin(\alpha) - cos(\delta)sin(\epsilon)cos(\alpha)d\alpha/dt
		\]
		\[
		\Rightarrow d\alpha/dt = (d\delta/dt = [(d\lambda/dt)\cancel{sin(\epsilon)\cos(\alpha)}]\frac{sin(\delta)sin(\epsilon)sin(\alpha)+cos(\epsilon)cos(\delta)}{cos(\delta)\cancel{sin(\epsilon)cos(\alpha)}}
		\]
		(we used equation (4))
		\[
		\Rightarrow d\alpha/dt = (d\lambda/dt)[tan(\delta)sin(\epsilon) sin(\alpha) + cos(\epsilon))]
		\]
		Actually the neighboring planet induced precession, introduces a constant rate of change of $\alpha$, lets call it -a
		\[
		d\alpha/dt = (d\lambda/dt)[tan(\delta)sin(\epsilon) sin(\alpha) + cos(\epsilon))] - a...........(5)
		\]
		
		Introducing new variables to remove constants:
		$m = d\lambda/dt cos(\epsilon) - a \, \& \, n = d\lambda/dt sin(\epsilon)$; gives (4) and (5) as
		
		
		\[
		d\alpha/dt = m + nsin(\alpha)tan(\delta) 
		\]
		\[
		d\delta/dt = ncos(\alpha)
		\]
		As it turns out to the lowest order in O(T) [$T = \frac{t-2000}{100}$]
		\[
		m = 1.2812 T \degree \, n = 0.5568 T \degree 
		\]
		For Proxima Centauri in epoch J2010:\\
		\[ T = 0.1 \Rightarrow m = 0.1281 \degree \, n = 0.0557 \degree\]
		\[
		\delta \alpha_p = 0.1281 +  nsin(217.429\degree)tan(-62.7128\degree) \approx 0.1937\degree =\delta \alpha_P
		\]
		Converting to hours:
		\[
		\boxed{\delta \alpha_P = (0.1937/15 *3600) ^s \approx 46.49^s}
		\]
		Similarly,
		\[
		\boxed{\delta \delta_p = ncos(217.429\degree) \approx -0.04423\degree \approx -2'39.23'' }
		\]
		Hence the precessed epoch J2010 coordinates for Proxima Centauri:
		\[
		(\alpha_P,\delta_P)_{precessed} = (14^h29^m40.95^s + 46.49^s, -62\degree40'46.1'' - 2'39.23'')
		\]
		\[
		\Rightarrow \boxed{(\alpha_P,\delta_P)_{precessed} \approx (14^h 30^m 29.4^s , -62\degree43'25.2'' )}
		\]
		
		\section*{\color{teal} Question 2- Blackbody Radiation}
		
		\textbf{Reference} :\url{https://edisciplinas.usp.br/pluginfile.php/48089/course/section/16461/qsp_chapter10-plank.pdf}

		\subsection{Specific intensity}
		
		The number of modes for a photon with frequency $\nu$ $\rightarrow$ $\nu + d\nu$ in a box of volume V($L^3$), i.e density of states(with dispersion relation E = pc = h$\nu$):\\
		The wavelenghth of photon would be quantised, since only n$\lambda$/2 = L would be resonant to the cavity dimensions.\\ 
		So \(p_{x,y,z} = hn_{x,y,z}/2L \)
		Let Photons with momenta components \(p_{x,y,z} + dp_{x,y,z} = h (n_{x,y,z} + \, d n_{x,y,z})/2L\)\\
		%Similarly for y and z; giving \(dp_xdp_ydp_z = dn_xdn_ydn_z L^3/8 \Rightarrow 4\pi p^2 dp = d\) 
		Hence the number of photons with absolute momenta between $|p|\rightarrow|p|+d|p|$ (momentum space volume = $4\pi p^2 d|p|$) , where $|p| = \sqrt{p_x^2 + p_y^2 + p_z^2}$ would be of the order $4 \pi \, \text{max}\,({n_{x,y,z}}^2) dn_{x,y,z} = dN (say)$ 
		\[
		dN (h/2L)^3 = (2) \times (1/8) \times 4 \pi p^2 dp 
		\](extra factor of two for polarisation states)(factor of 1/8 since n$_{x,y,z}$ can take only positive values)
		
		\[
		dN = \frac{8\pi V}{h^3} (E/c)^2 d(E/c)
		\]
		\[
		dN = \frac{8\pi V}{c^3} \nu^2 d\nu 
		\]
		where we used the dispersion relation E =pc =h$\nu$\\
		Now the photon mode occupancy factor is calculated by the Bose-Einstein Partition function (photon number unrestricted)
		\[
		\mathcal{Z}_{BE} = \Sigma_{N=0}^{\infty} exp(-\beta nh\nu) = \frac{1}{1-\exp(-\beta h \nu)} 
		\]
		which is like a normalisation factor. Hence ,Occupancy fraction(probability) : 
		\[
		P_{BE} = \frac{exp(-\beta h\nu)}{1-exp(-\beta h \nu)} = \frac{1}{exp(\beta h \nu) -1}
		\]
		Thus number of photons in modes $\nu \rightarrow \nu + d\nu$ in a given volume V per unit frquency,
		\[
			n_\nu = \frac{dN}{Vd\nu} = \frac{8\pi}{c^3} \nu^2 \frac{1}{\exp(\beta h \nu)-1}
		\]
		Hence the Energy density per unit volume per unit frequency:
		\[
			u_\nu = n_\nu h\nu = \frac{8\pi h}{c^3} \nu^3 \frac{1}{\exp(\beta h \nu)-1}
		\]
		Spectral intensity(for all solid angle):
		\[
			\boxed{u_\nu \times c/4\pi = B_\nu = \frac{2h}{c^3} \nu^3 \frac{1}{\exp(\beta h \nu)-1}}
		\]
		
		\subsection{Spectral intensity curves do not intersect}
		
		For \( T_2 > T_1 \) and $\nu > 0 \Rightarrow $ $\nu/T_2 < \nu/T_1$: strict inequality.\\
		Now since exponential function is a strict monotone:
		\[
		exp(\nu/T_2) -1 < exp(\nu/T_1)-1
		\]
		Now same as first argument:
		\[
		\frac{1}{exp(\nu/T_2) -1} \nu^3 > \frac{1}{exp(\nu/T_1)-1} \nu^3
		\]
		Hence at two strictly different temperatures the spectral intensity factors are strictly different. Since, $\nu$ in the above argument was arbitrary positive definite quantity, this argument holds for all \(\nu\). So the spectral intensity curve do not intersect throughout their domain. 
		
		\subsection{Maxima of Planck function}
		
		Taking the derivative of spectral function (modulo constants) and setting it equal to zero:
		\[
			d \frac{\nu^3}{exp(\beta h \nu) - 1} \vline_{v_{max}}= 0			
		\]
		\[
			\Rightarrow \frac{3\nu_M^2}{exp(\beta h \nu_M) - 1} - \frac{\nu_M^3exp(\beta h \nu_M)(\beta h)}{(exp(\beta h \nu_M) - 1)^2} = 0
		\]
		\[
			[\frac{\nu_M^2}{exp(\beta h \nu_M) - 1}][3-\frac{\nu_M exp(\beta h \nu_M)\beta h}{exp(\beta h \nu_M) - 1}] = 0 
		\]
		Since $\nu_M \neq 0 $ so left bracket can be ignored:
		\[
			(3-x)exp(x) = 3
		\]where x = $\beta h \nu_M$
		Solving this transcendental equation numerically on mathematica gives:
		\[
			 x = 2.82 \Rightarrow \boxed{h \nu_M = 2.82 K_B T}
		\] 
		\subsection{Total energy density over all frequencies}
		\[
			u = \int u_\nu d\nu = \int_0^{\infty} \frac{8\pi h}{c^3} \nu^3 \frac{1}{\exp(\beta h \nu)-1} d\nu
		\]
		variable transform : $\beta h\nu = x$
		\[
			 \frac{8\pi h}{c^3}(\frac{K_B T}{ h})^4 \int_0^{\infty} x^3  \frac{1}{\exp(x)-1} dx
		\]
		The integral is a standard integral and its value is $\pi^4/15${\color{red}\Huge (appendix)}
		
		Thus,
		\[
			\boxed{u_\nu = \frac{8\pi^5K_B^4}{15(ch)^3} T^4 = a T^4}
		\]
		Hence,
		\[
			\boxed{a = \frac{8\pi^5K_B^4}{15(ch)^3} = 7.536 \times 10^{-15} erg \, cm^{-3} \, K^{-4}}
		\]
		\subsection{Stefan Boltzmann Law}
		
		The energy radiated by black body per per unit forward area per unit time:
		\[
			F = \int F_\nu d\nu =\int B_\nu cos\theta d\Omega  = \int B_\nu \int_{0}^{\pi/2} cos\theta sin\theta d\theta \int_{0}^{2\pi} d\phi = \pi\int B\nu 
		\]
		Looking at the radiation recieved only in the forward direction (=1/2 =hemisphere)
		\[
			= \pi \times c/4\pi \int u_\nu d\nu = \boxed{ac/4 \times T^4 = \sigma T^4 =F}
		\]
		Half the recieved photons go in the perpendicular direction due to isotropy.(1/2 $\times$ 1/2 =1/4)
		Hence:
		\[
			\boxed{\sigma = ac/4} = \frac{7.536 \times 10^{-15} . 3 \times 10^{10}}{4} =\boxed{ 5.67 \times 10^{-5} erg \, cm^{-2} \, s^{-1} \, K^{-4}=\sigma}
		\]
		
		\subsection{Entropy of Blackbody Radiation}
		
		\textbf{Reference}:Harvey S. Leff; Teaching the photon gas in introductory physics. Am. J. Phys. 1 August 2002; 70 (8): 792–797\url{https://doi.org/10.1119/1.1479743}\\
			
		The RMS velocity of the photons in the forward direction(x) is given by:
	
		\[
			\bar{c_x^2} = \int \frac{Vn(c_x)}{N} c_x^2 dc_x =1/3 c^2
		\]
		(due to isotropic of radiation, where $(c_x)$ is the velocity distribution)
		
		Pressure exerted on wall of area A by photons of average energy E:
		\[
			P = (F/A) . N = \langle \frac{1}{A\triangle t}\frac{E}{c}\frac{c_x}{c} . (Ac_x\triangle t)n(c_x)\rangle
		\] 
		where, Impulse due to photons = change in x-momentum (c$_x$/c fraction of E/c=p)\\
		Also, number of photons colliding = n(c$_x$)$\times$ volume of photons aproachable in time $\triangle t(=A c_x\triangle t)$ 
		
		\[
			P = E/c^2 \langle c_x^2 n(c_x)\rangle \rightarrow E/c^2 . N/V \bar{c_x^2} = (E/c^2) (N/V) (c^2/3) =EN/3V = U/3V
		\] 
		\[
			\Rightarrow U=3PV
		\]
		is the equation of state of photon gas\\
		The entropy:
		\[
			\boxed{S = \frac{1}{T}(U+PV) = \frac{1}{T}(4/3 U) = \frac{4}{3} aVT^3}
		\]
		
		\section{\color{teal}Question 3- Eye and Radiation}
		
		\subsection{Energy of photon gas resonant in the eye}
		\[
			U = aVT^4
		\]
		where \(V = \frac{4}{3}\pi r^3 = 14.1372 \, cm^3\) for the given dimension.
		T = body temperature = \(37 \degree C =311.15 K\)
		\[
			\Rightarrow U = 7.536 \times 10^{-15} \times 14.1372 \times 311.5^4 \boxed{\approx 0.001 \, ergs +U_{eye}}
		\]
		\subsection{Bulb radiation energy received by eye}
		\[
			\boxed{E/t = \frac{Power}{4\pi d^2} \, dA_{pupil} = \frac{10^9 erg/sec}{4\pi 100^2} 0.1 \, cm^2 = 795.775 \, erg \, s^{-1} } 
		\]
		Over a second the \(E = 795.78 \, ergs \approx 8 \times 10^5 E_{closed \, eye}\)
		
		\subsection{Darkness in closed Eye}
		
		Using Wien's displacement law, the peak energy density in our eye is due to light correspnding to wavelength 
		\[
			\boxed{\lambda_M = b/T = \frac{2898 \mu m K}{311.5 K} \approx 9.3 \mu m}
		\]
		Which lies in the mid infrared region to which our eye is insensitive, and hence its dark when we close our eyes.
		
		
		
	\end{document}